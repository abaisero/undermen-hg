\documentclass[10pt,fleqn]{article}

\usepackage{amsmath,amsfonts,amssymb,amsthm}
\usepackage{paralist}
\usepackage{todonotes}

\usepackage{tikz}
\usepackage{tikz-qtree}
\usetikzlibrary{arrows}

% TODO complete todo commands..

%%%%%%%%%%%%%%%%%%%%%%%%%%%%%%%%%%%%%%%%%%%%%%%%%%%%%%%%%%%%%%%%%%%%%%%%%%%%%%%%
%% Document options
%%%%%%%%%%%%%%%%%%%%%%%%%%%%%%%%%%%%%%%%%%%%%%%%%%%%%%%%%%%%%%%%%%%%%%%%%%%%%%%%

\newcommand{\nosubsubcounter}{\setcounter{secnumdepth}{2}}

%%%%%%%%%%%%%%%%%%%%%%%%%%%%%%%%%%%%%%%%%%%%%%%%%%%%%%%%%%%%%%%%%%%%%%%%%%%%%%%%
%% Custom commands
%%%%%%%%%%%%%%%%%%%%%%%%%%%%%%%%%%%%%%%%%%%%%%%%%%%%%%%%%%%%%%%%%%%%%%%%%%%%%%%%

% TikZ
\newcommand{\inputTikZ}[1]{
  \input{{#1}.tex}
}

% Custom TODO commands
\newcommand{\intodo}[1]{
  \todo[inline]{{#1}}
}
\newcommand{\nytodo}[1][written]{
  \todo[inline]{Not yet {#1}}
}
\newcommand{\todopage}{
  \ifnum\value{@todonotes@numberoftodonotes}>0
    \newpage\listoftodos
  \fi
}

%%% Environments

% Inline list
\newenvironment{inlist}
{\begin{inparaenum}[\itshape a\upshape)]}
{\end{inparaenum}}

% Example environment, section numbering
\newtheorem{example}{Example}[section]

%%% Reference commands
\newcommand{\EX}[1]{Ex.~\ref{ex:#1}}
\newcommand{\EXAMPLE}[1]{Example~\ref{ex:#1}}
\newcommand{\EQ}[1]{\eqref{eq:#1}}
\newcommand{\EQUATION}[1]{Equation~\EQ{#1}}

\newcommand{\SECTION}[1]{Section~\ref{sec:#1}}
\newcommand{\SEC}[1]{Sec.~\ref{sec:#1}}

\newcommand{\FIG}[1]{Fig.~\ref{fig:#1}}
\newcommand{\FIGURE}[1]{Figure~\ref{fig:#1}}
\newcommand{\SUBFIG}[1]{\subref{fig:#1}}

\newcommand{\TABLE}[1]{Table~\ref{tab:#1}}

\newcommand{\ITEM}[1]{(\ref{item:#1})}

%%% Textual modifiers
\newcommand{\s}[1]{``#1''}

%%% Common texts

\newcommand{\etal}{et al}
\newcommand{\ie}{i.e.}
\newcommand{\eg}{e.g.}
\newcommand{\etc}{etc}
\newcommand{\etcdots}{\etc\ldots}
\newcommand{\dash}{--}

%%%%%%%%%%%%%%%%%%%%%%%%%%%%%%%%%%%%%%%%%%%%%%%%%%%%%%%%%%%%%%%%%%%%%%%%%%%%%%%%
%% Math Environment Common Commands
%%%%%%%%%%%%%%%%%%%%%%%%%%%%%%%%%%%%%%%%%%%%%%%%%%%%%%%%%%%%%%%%%%%%%%%%%%%%%%%%

% Common operators
\DeclareMathOperator*{\argmin}{arg\,min}
\DeclareMathOperator*{\argmax}{arg\,max}
\newcommand{\abs}[1]{\left|\,#1\,\right|}
\newcommand{\dotp}[2]{\langle\,#1,#2\,\rangle}

% Common symbols
\newcommand{\Rs}{\mathbf{R}}
\newcommand{\Ss}{\mathcal{S}}
\newcommand{\Xs}{\mathcal{X}}
\newcommand{\Ys}{\mathcal{Y}}
\newcommand{\defeq}{\ensuremath{\mathrel{\stackrel{\mathrm{def}}{=}}}}

% Symbol modifiers
\newcommand{\vv}[1]{\boldsymbol{#1}}
\newcommand{\alt}[1]{\tilde{#1}}

% Text in Formula commands
\newcommand{\st}{\text{ s.t. }}


\nosubsubcounter


\title{Notes}
\author{Andrea Baisero}

\begin{document}
\maketitle

\section{Rules}

\subsection{Food and Winning}

Each player begins the game with $300(P-1)$ units of food, where $P$ is the
number of players.

If after any round you have zero food, you will die and no longer be allowed to
compete. All players who survive until the end of the game will receive the
survivor's prize.

The game can end in two ways. After a large number of rounds, there will be a
small chance each additional round that the game ends. Alternatively, if there
is only one person left with food then the game ends. In each case, the winner
is the person who has the most food when the game ends.

The game is designed to encourage both cooperation and competition between
players at different points in the game. Determining the right balance leads to
a winning strategy.

\subsection{Hunts}

Each round is divided into hunts. A hunt is a game played between you and one
other player. Each round you will have the opportunity to hunt with every other
remaining player, so you will have $P-1$ hunts per round, where $P$ is the
number of remaining players.

In a hunt you can make the decision to hunt (become a hunter) or feign sickness
and not hunt (become a slacker). Hunting is an active activity, so you will eat
$6$ units of food if you decide to hunt, and $2$ units of food if you decide not
to hunt. If both you and your partner do not hunt, then you get $0$ food from
the hunt. If only one of you decides to hunt then the hunt returns $6$ units of
food.  If both of you decided to hunt then the hunt returns $12$ units of food.
The total food return of the hunt is split evenly between you and your partner.
So, for example, if both of you decide to hunt, your individual net food
gain/loss from the hunt would be $12/2 - 6=0$ units, whereas if both of you
decided not to hunt your individual net food gain/loss is $0/2 - 2= -2$ food.

\subsection{Reputation}

Each player has a reputation $R$, which is defined as $R=\frac{H}{H+S}$, where
$H$ is the number of times you chose to hunt and $S$ is the number of times you
chose to slack (not hunt), both counted from the beginning of the game. Each
player begins with a reputation of $0$.

Each round, your tribe can save some time and hunt extra food if enough people
opt to hunt. Each round, a random integer m, with $0<m<P(P-1)$ will be chosen.
If the total number of hunters is greater than or equal to m for the upcoming
round, then everyone in the tribe gets $2(P-1)$ extra units of food after the
round, which means that on average everyone gets 2 units of food per hunt.
Before each round, you will find out the value of m.

Before each round you will be given: which number round you are in (from $1$ to
$\infty$), your food, your reputation, $m$, and an array with the reputation of
each remaining member of the tribe when the round started. The location of each
player in the array will be randomized after each round. You should decide your
moves for that round for each player you will play against, and return an array
of your decisions. After each hunt you will be given the food earned from each
hunt you played.

\subsection{Submission guidelines}

The logic of your code and what game theory analyses you apply for the various
steps must be either laid out as comments in the code or in a separate
document. The algorithm (and separate document, if any) should be submitted by
August 8, 2013 through your Brilliant account. Algorithms will be run
tournament-style to determine the winner. The finalist teams will then have a
short video interview with Brilliant staff, to verify that their work is their
own.

Prizes for algorithms eligible for the final game will be awarded at two
levels:

Grand Prize Winners \dash\ $5$ teams will receive the Grand Prize of \$1000.
The winning algorithm is guaranteed to receive the Grand Prize. $4$ other winners
will be selected based on the performance of their algorithm, logic, game
theory analysis, and presentation.  Finalists \dash\ Members of teams whose
algorithm survives the game will receive an \s{I survived} t-shirt.

\section{Recap of Rules}

\subsubsection{Terminology}
\begin{itemize}
  \item \emph{slot} \dash\ One single hunting session with one single opponent.
  \item \emph{round} \dash\ A collection of hunting sessions, each with a
    different opponent.
  \item \emph{m-award} \dash\ The award everyone gets when enough people
    cooperate.
\end{itemize}

\subsubsection{Annotation}

\begin{tabular*}{\textwidth}{cl}
  \toprule
  Symbol & Meaning \\
  \midrule
  $A, A'$ & Actions of my agent and opponent agent respectively. \\
  \midrule
  $U(A, A'), U(A', A)$ & Utility function for my agent and opponent agent
  respectively. \\
  \midrule
  $T$ & Current number of rounds. Starts with $1$. \\
  \midrule
  $p_t$ & Number of slots at round $t$. \\
  \midrule
  $P_t$ & Number of total slots from time $1$ until round $t$ excluded. \\
  \midrule
  $h_t, s_t$ & Number of hunts/slacks (respectively) my agent performs at round
  $t$. \\
  \midrule
  $H_t, S_t$ & Number of total hunts/slacks (respectively) from round $1$ until
  round $t$ excluded. \\
  \midrule
  $R_t, R'$ & My reputation (at round $t$) and the opponent's reputation
  respectively. \\
  \midrule\
  $pc, pc'$ & Probability of my agent and opponent agent cooperating
  respectively. \\
  \bottomrule\
\end{tabular*}

\intodo{FER: Why are there two elements players/rounds associated with one
variable $p_t$ (or $P_t$)? I understand that the number of players is not the
same as the number of rounds. The game is divided in rounds, and during each
round each player participates in P-1 hunts.

BAIS: I probably didn't write that clear enough, I tried rephrasing it.
Basically $p_t$ represents the number of slots in one specific round, while
$P_t$ sums up all the slots that have happened in the previous rounds. $P_t$
allows us to determine exactly how many ways an opponent has hunted or slacked,
given his reputation (not sure if this is helpful), but even better, it gives a
measure of how many players are dying at each round.}

\subsubsection{Some Relationships}
\begin{align*}
  P_t &= P_{t-1} + p_{t-1} = \sum_{\tau = 1}^{t-1} p_\tau \\
  H_t &= H_{t-1} + h_{t-1} = \sum_{\tau = 1}^{t-1} h_\tau \\
  S_t &= S_{t-1} + s_{t-1} = \sum_{\tau = 1}^{t-1} s_\tau \\
  P_t &= H_t + S_t \\
  p_t &= h_t + s_t \\
  R_t &= \frac{H_t}{P_t} \\
  \intertext{The next one is important for reputation balancing ($h_t$ would be
  our control variable),} \\
  R_{t+1} &= \frac{H_{t+1}}{P_{t+1}} \\
          &= \frac{H_t+h_t}{P_t+p_t} \\
          &= \frac{R_t}{R_t} \frac{H_t+h_t}{P_t+p_t} \\
          &= R_t \frac{P_t (H_t+h_t)}{H_t (P_t+p_t)} \\
          &= R_t \frac{P_t}{P_t + p_t} \frac{H_t + h_t}{H_t} \\
          &= R_t \frac{P_t}{P_t + p_t} + R_t \frac{P_t}{P_t + p_t} \frac{h_t}{H_t}
\end{align*}

\intodo{FER: Is it possible that the first sum is until $\tau=t-1$? If so, it may
apply for the others as well.

BAIS: Yes, you are right.. I fixed it.}

\section{Strategy}

\subsection{Mixed Strategies}
\emph{Should we make deterministic decisions?}

Without loss of generality, we should probably start working directly using
mixed strategies, where we assign a probability of cooperation to both our agent
and the opponent's.

\intodo{FER: This is a good idea, I agree with it completely for the moment.

BAIS: Although, to say the truth I am coming to believe that there is really no
reason to include stochasticity in our own actions, given that people will
hardly build a profile on us.. This connects with the fact that it doesn't
matter with whom we hunt or slack, but only how many times we do so..
}

\subsection{Opponent strategy}

A naive idea is to have the opponent's probability of cooperation approximated
as his own reputation, \ie\ $pc' \approx R'$. This is very simple, but
unfortunately it is also too suboptimal. If we take this assumption, then the
optimal strategy becomes to always deflect.

More generally, we should aim at modelling his probability of cooperation as a
function of not only his reputation, but also mine, \ie\ $pc' \approx f(R',
R_t)$.

\subsection{Utilities}

\subsubsection{Utility Table}
\begin{equation*}
  U =\,
    \begin{tabular}{|c||c|c|}
      \hline
      $A$ \textbackslash\ $A'$ & H & S \\
      \hline\hline
      H & $0$ \textbackslash\ $0$ & $-3$ \textbackslash\ $+1$ \\
      \hline
      S & $+1$ \textbackslash\ $-3$ & $-2$ \textbackslash\ $-2$ \\
      \hline
    \end{tabular}
\end{equation*}

\subsubsection{Expected Utility}

Let's try to analyze our expected utilities, taking the naive approximation of
$pc'$.

\begin{align*}
  E[U(A, A') / A = H] &= E[U(H, A')] \\
                      &= U(H, H) pc' + U(H, S) (1-pc')\\
                      &= 0 pc' + -3 (1-pc') \\
                      &= -3 + 3 pc'
\end{align*}
\begin{align*}
  E[U(A, A') / A = S] &= E[U(S, A')] \\
                      &= U(S, H) pc' + U(S, S) (1-pc') \\
                      &= 1 pc' + -2 (1-pc') \\
                      &= -2 + 3 pc'
\end{align*}
Notice how the second is always higher than the first. This simply confirms that
in the non-iterated prisoner's dilemma (or in the case we assume the opponent
doesn't base his strategy on our own reputation) the best strategy is always to
slack.

When (if) we manage to model $pc'$ also as a function of our own reputation, we
might try this approach again..

\subsection{Relative interpretation of Utilities}
\emph{When is it worth it to risk by cooperating?}

It can be argued that it is mostly unimportant exactly how much we win or lose
in absolute terms, but rather it is important only how much we win or lose \wrt\
all the other possible outcomes.

This is pragmatically true only as long as we can avoid having very low stacks
of food, in which the importance of considering the absolute terms arises.
Consider, in the game at hand, that your action may change the outcomes either
from $1$ to $-2$ or from $0$ to $-3$. In both cases, the relative change is
always $-3$, and as such the analysis is simplified. However, if you actually do
have few items of food left, then the difference between the absolute terms
actually starts to gain importance. However, I find two reasons why this can be
largely ignored:
\begin{inlist}
  \item If you have so few items of food left, you are practically already
  doomed. You will die inevitably unless you find a very long streak of
  collaborators in front of you, a very unlikely event. Even so, you will only
  either stay at the same food level (if you collaborate with them), or gain $1$
  very puny food item, at the cost of decreasing your reputation for each puny
  gain.
  \item The game starts with a food stack in an order of magnitude which I
  estimate to be between $10^4$ and $10^6$. As such, the absolute difference of
  $1$ between the outcomes is largely ignorable.
  \item More importantly, if we ever do get at such low levels of food, it means
  that our strategy was a losing one to start with, and that can not be
  attributed to the change of P.O.V from absolutist to relativist utility
  interpretations.
\end{inlist}

So let's assume that we are only interested in a relative interpretation of
utilities. As such, the game's utilities can be translated as:
\begin{itemize}
  \item For each round, we lose $2$ items of food consistently. This is the term
  which can be pragmatically ignored.
  \item We lose $1$ extra item of food per session if we choose to cooperate.
  \item We gain $3$ items of food per session if our opponent chooses to
  cooperate.
\end{itemize}

In a table representation, the relativist P.O.V of utility simply means that we
can adjust all the utilities by adding/subtracting a constant:
\begin{equation*}
  U =\,
    \begin{tabular}{|c||c|c|}
      \hline
      $A$ \textbackslash\ $A'$ & H & S \\
      \hline\hline
      H & $+2$ \textbackslash\ $+2$ & $-1$ \textbackslash\ $+3$ \\
      \hline
      S & $+3$ \textbackslash\ $-1$ & $0$ \textbackslash\ $0$ \\
      \hline
    \end{tabular}
\end{equation*}

It is thus easy to postulate that any agent should only be interested in
adopting strategies which include cooperation component if and only if these
allow the agent to receive at least $1$ cooperation for every $3$ which he gives
away. This aspect is fundamentally true, and should be included in the final
strategy.

If we port this in the framework of mixed strategies, then we can say that an
agent should adopt cooperation-inclusive strategies if and only if the
probability of receiving cooperation is at least one third of the probability of
giving out cooperation.
\intodo{Is this correct? Not sure.. Think more about it. UPDATE: pretty sure
it's not.}

If these guidelines are not guaranteed, then the agent performs best to simply
avoid all sorts of cooperation at all.


\subsubsection{Conclusion}
Choosing a policy which maximises the previous expectation leads to a static
policy of always slacking, which overall would lead to a consistent loss of 2
items of food for every round.

However, we will see that maybe we are not really using all the information we
have. Up to now, we have assumed that the opponent's action is only dependent on
his responsibility, but we have left out that his actions in front of \emph{us}
are very likely to depend on our own responsibility.

The next section will try to analyse why and how our reputation changes our
opponents' actions, and how to make use of this to increase our expected food
value.

\subsection{Reputation Balancing}
\emph{Why?}

We have already seen that slacking is the Nash Equilibrium solution, \ie\ the
local optimum when we try to maximize the expected utilities. Unfortunately,
this is far from a global optimum: Being this a prisoner dilemma problem, we
have that both players gain by cooperating, but only if they both cooperate.
This is the problem we are trying to address.

It is reasonable to assume that \emph{some} players will try to cooperate so
that they all gain by hunting together. If such a group forms, then I would
rather be in it than out of it. The intended goal is to avoid being classified
as a \s{slacker}: If you are a slacker, then everyone will slack against you.

The problem is that there is no means of communication with the community so as
to organize the up-coming of such a group: The only information we can exchange
is our reputation. The goal of reputation balancing should be to send a message
to your opponent: \s{look: you can trust me}.

On the other hand the creation of such a group opens ground for all sorts of
treason and backstabbing. Confirming yourself as a strong hunter is also not a
good strategy: If you are a compulsive hunter then other people will assume they
can slack their way into profit at your expenses. All in all, we just want to
keep our reputation just high enough for that to the message to pass. We gain
the maximum when we cooperate just enough to let the message pass. The question
now is: \emph{what is the threshold?}

One other interesting thing to consider is that the message we pass is only
effective if there is someone listening on the other end. If we are surrounded
by compulsive slackers, then there is little we can do but slack
ourselves\footnote{This is the reason why might want to keep a global reputation
measure at hand, but more on that later.}.

\subsubsection{What value to aim for?}

This is a very tough question.

Assuming we would simply want to avoid the opponent to predict our own
decision, then we would want to maximize the future entropy
\begin{align*}
  H^{entropy}(R_{t+1}) &= - R_{t+1} \log(R_{t+1}) - (1-R_{t+1}) \log(1-R_{t+1})
\end{align*}

To maximize, we simply aim at having $R_{t+1} = 0.5$, which is quite intuitive
considering our goal of avoiding possible predictions by our opponent.

However, there's still problems with this approach:
\begin{inlist}
  \item We are only maximizing entropy under the assumption that your opponent
  takes your reputation as a naive approximation of your probability of
  cooperating.
  \item What are exactly the advantages of maximizing your opponent's entropy?
  This is not poker, if two people want to cooperate it is not enough to come
  out as a guy who might back-stab you $50\%$ of the times. I strongly suspect
  that the value of reputation we should aim for is higher than $.5$.
\end{inlist}

\subsubsection{How?}

This highly depends on what the relationship is.

\subsection{Public Goods}
\emph{Are Public Goods good or bad?}

\begin{quotation}
  \s{In many realistic situations there are benefits to cooperation that accrue
  to all players, but only if a certain number of players cooperate. These
  benefits are called public goods. A typical public good is a community
  playground built by donations. The playground can be built only if enough
  people help, but the playground benefits everyone equally. Public goods tend
  to increase the chance that people will cooperate.}
\end{quotation}
\hfill from \emph{Training Exercise}

The only serious observation I can make regarding the issue of public goods is
that rich people don't want to achieve public goods, whereas poor people do. The
reason is simple: Public goods affect everyone positively. As such, rich people
would be losing their dominance against poorer people, and the other way round.
For example, consider the difference between two players who have $100$ and $5$
respectively and who have $900$ and $805$ respectively. In the former case, the
richer guy is dominating; in the latter, the dominance is daft.

\subsection{Tit-for-tat}
\emph{What is it? Can it be applied?}

\begin{quotation}
Axelrod discovered that when these encounters were repeated over a long period
of time with many players, each with different strategies, greedy strategies
tended to do very poorly in the long run while more altruistic strategies did
better, as judged purely by self-interest. He used this to show a possible
mechanism for the evolution of altruistic behaviour from mechanisms that are
initially purely selfish, by natural selection.

The winning deterministic strategy was tit for tat, which Anatol Rapoport
developed and entered into the tournament. It was the simplest of any program
entered, containing only four lines of BASIC, and won the contest. The strategy
is simply to cooperate on the first iteration of the game; after that, the
player does what his or her opponent did on the previous move. Depending on the
situation, a slightly better strategy can be "tit for tat with forgiveness."
When the opponent defects, on the next move, the player sometimes cooperates
anyway, with a small probability (around 1–5%). This allows for occasional
recovery from getting trapped in a cycle of defections. The exact probability
depends on the line-up of opponents.
\end{quotation}
\hfill from \emph{Wikipedia}

The tit-for-tat is a strategy in this you start out by being nice (\ie\
cooperating), and after that you \s{repay} the opponent with whatever the same
action he has used against you. Practically, \s{an eye for an eye}. Classically,
this requires you to know how your opponent has acted against you, so that you
can repay you. In this instance of the game we can not identify users between
successive rounds, and so it is impossible to repay each exact opponent with his
own action. However, we can still actuate a \s{tit for tat} in which we just
\s{repay} different people with the same actions we were dealt in the previous
turn. Basically, this simply gives us a guideline of how many hunts/slacks to
deal in the next turn given the ones we have received in the previous turn.

\subsubsection{First Round}
Classically, the \s{tit for tat} strategy starts by cooperating blindly. Is this
a good idea? What do we expect to happen in the best, average and worst cases?

In the best case, every opponent starts by collaborating. If that is so, then
you will neither gain nor lose any points (not counting a possible m-award).

In the worst case, every opponent starts by deflecting. In that case, you will
lose $3*(P-1)$ points, which amount to $1\%$ of your starting points.

In the average case, you will lose something between $[0, 1]\%$ points, which is
not bad if we are actually going to adopt this strategy.

\subsubsection{Relation with Global Reputation}
I suspect heavily that actuating this strategy is equivalent to simply
maintaining our reputation equal to a global reputation which is simply the mean
of all the reputations. The benefits of this are still up to debate.

UPDATE:\ The result is actually not just to chase the average reputation.. No,
no..  Rather, it makes my reputation tend to the average number of times, in
percentage, that we receive hunts.

\subsection{TODO}
\emph{What is missing?}

The most crucial thing which is still missing is how to model how our reputation
influences the opponent's chance of cooperation..

\section{Final Strategy}

Given the above considerations, the idea seems to arise that the only important
thing to decide at each round is $h_t$. Given the relativist P.O.V of utility, it
doesn't matter exactly with \emph{who} we collaborate or not, but only \emph{how
many times} we do. This is enforced and assured also by the fact that our
opponents can not identify us, and never even receive information about which
action ours was.
\intodo{FER: You actually get to know the rewards of each of the hunts you have
participated in, thus the actions of the opponents.

BAIS: Yes indeed. They do get to know which action we chose in any specific
turn, however profiling remains virtually impossible because of the player
shuffling, unless the number of players reduces to $100$ or less, which I doubt
it will, given the presence of m-award.

FER: We agree on this. Although it is in principle possible to profile the players,
the shuffling makes it unfeasible or, at least, a hard task. I think we can safely
assume people will not be doing it.

How do you infer from the presence of the m-award that there will be more than
$100$ players?
}

Thus the idea of the final algorithm is the following:
\begin{itemize}
  \item Determine what kind of reputation maximizes the number of collaborations
  you receive, and determine if that kind of reputation is too expensive to get
  (more than 3 of your collaborations to get 1 of theirs).
  \item Determine how many collaborations you need to give out to achieve that
  reputation.
  \item Scatter the collaborations across the opponents. There is only one
  aspect \wrt\ which it does matter to who you do or you do not collaborate: Try
  to reward the slackers rather than the hunters, because the (successful)
  hunters are likely to already have more food than the slackers. To
  successfully undermine the opposition, only give advantage to who doesn't pose
  a threat\footnote{Funny observation: If too many people reason like me,
  then the slackers would totally win. However, I doubt that will be the
  case for a couple reasons..}.
\end{itemize}

\intodo{FER: I am not sure about the last point. I believe that the presence of
slackers would tend to make more players slackers. Too many slackers is not
something good, every player gradually loses food in that situation. In order to
avoid this scenario, at the beginning the player tries to gain reputation, ensuring
some level of cooperation (hunt/hunt) among the players. Once this is achieved, the
focus can be turned into keeping the reputation within an uncertain range (opponents
do not clearly distinguish us as slackers or hunters), while keeping high the level
of food (very vague how).

BAIS: We better speak about this more directly.. I would have too many things to
write on this alone \smiley}

\subsubsection{Methodology}
\emph{How should one proceed methodologically?}

It is quite hard, virtually impossible, to develop an optimal strategy for this
challenge; there are too many variables to consider, too many possibilities. One
way to go through the process is to start considering different types of
assumptions which will simplify the remaining part of the problem. The more an
assumption is \s{reasonable}, the more the relative strategy is an approximation
of the optimal strategy. Different assumptions will lead straightforwardly
towards different strategies, so I think we should formulate them in parallel.

Some of the assumptions one might formulate are the following:

\begin{tabular}{ccp{.65\textwidth}}
  \toprule\
  Code & Requisites & {\bfseries Assumption.} Reason \\
  \midrule\
  A & \dash\ & {\bfseries It is better to only slack.} After all, slacking is
  the local optimal in one single round. Obsolete. \\
  \midrule\
  B & $\lnot$A & {\bfseries It is better to collaborate sometimes.} if that
  convinces other players to collaborate with you. \\
  \midrule\
  C & \dash\ & {\bfseries Most of the community will try to punish slackers.} \\
  \midrule\
  D & \dash\ & {\bfseries There will always be more than 100 players.} We can
  assume that (or not) depending on how we value the m-award to be important for
  the survival of many people. \\
  \midrule\
  E & D & {\bfseries No-one is able to track agent identities.} Reasonable
  unless there are few players ($<100$) left. \\
  \midrule\
  F & E & {\bfseries Out opponent bases his decision only depending on his, mine,
  and the global reputation.} These three parameters are the only source of info
  left on other players. \\
  \midrule\
  G & \dash\ & {\bfseries Relativistic nature of utilities.} emerges naturally.
  Implies that we lose 1 point for each hunt we do, and gain 3 points for each
  hunt our opponent does. \\
  \midrule\
  H & G & {\bfseries It is better to only slack (PART 2) if you receive less
  than 1 collaboration for each 3 that you give.} \\
  \midrule\
  I & \dash\ & {\bfseries Many players will not recognize G.} Hard to justify or
  negate. \\
  \midrule\
  J & G & {\bfseries The main (only) strategic decision to make is how many
  times to hunt.} If you only consider the goal of maximizing your own income.
  \\
  \midrule\
  K & C, E & {\bfseries It is better to slack against a hunter and to hunt
  with slackers.} If C and E are true, then we might as well decide to slack
  (\ie\ penalize) the hunters (the strong competition) and hunt (\ie\ promote)
  the slackers (the weak competition) rather than the other way round. \\
  \midrule\
  L & \dash\ & {\bfseries It is better to always slack (PART 3) if enough people
  understand J and agree on K.} Because slacking will be enforced anyway rather
  than punished. \\
  \midrule\
  M & \dash\ & {\bfseries Tit-for-tat can be applied here too!} Niiiice! But
  what does it imply? It probably equates to maintaining your reputation equal
  to some measure of global reputation. \\
  \bottomrule\
\end{tabular}

\end{document}

