\documentclass[10pt,fleqn]{article}

\usepackage{amsmath,amsfonts,amssymb,amsthm}
\usepackage{paralist}
\usepackage{todonotes}

\usepackage{tikz}
\usepackage{tikz-qtree}
\usetikzlibrary{arrows}

% TODO complete todo commands..

%%%%%%%%%%%%%%%%%%%%%%%%%%%%%%%%%%%%%%%%%%%%%%%%%%%%%%%%%%%%%%%%%%%%%%%%%%%%%%%%
%% Document options
%%%%%%%%%%%%%%%%%%%%%%%%%%%%%%%%%%%%%%%%%%%%%%%%%%%%%%%%%%%%%%%%%%%%%%%%%%%%%%%%

\newcommand{\nosubsubcounter}{\setcounter{secnumdepth}{2}}

%%%%%%%%%%%%%%%%%%%%%%%%%%%%%%%%%%%%%%%%%%%%%%%%%%%%%%%%%%%%%%%%%%%%%%%%%%%%%%%%
%% Custom commands
%%%%%%%%%%%%%%%%%%%%%%%%%%%%%%%%%%%%%%%%%%%%%%%%%%%%%%%%%%%%%%%%%%%%%%%%%%%%%%%%

% TikZ
\newcommand{\inputTikZ}[1]{
  \input{{#1}.tex}
}

% Custom TODO commands
\newcommand{\intodo}[1]{
  \todo[inline]{{#1}}
}
\newcommand{\nytodo}[1][written]{
  \todo[inline]{Not yet {#1}}
}
\newcommand{\todopage}{
  \ifnum\value{@todonotes@numberoftodonotes}>0
    \newpage\listoftodos
  \fi
}

%%% Environments

% Inline list
\newenvironment{inlist}
{\begin{inparaenum}[\itshape a\upshape)]}
{\end{inparaenum}}

% Example environment, section numbering
\newtheorem{example}{Example}[section]

%%% Reference commands
\newcommand{\EX}[1]{Ex.~\ref{ex:#1}}
\newcommand{\EXAMPLE}[1]{Example~\ref{ex:#1}}
\newcommand{\EQ}[1]{\eqref{eq:#1}}
\newcommand{\EQUATION}[1]{Equation~\EQ{#1}}

\newcommand{\SECTION}[1]{Section~\ref{sec:#1}}
\newcommand{\SEC}[1]{Sec.~\ref{sec:#1}}

\newcommand{\FIG}[1]{Fig.~\ref{fig:#1}}
\newcommand{\FIGURE}[1]{Figure~\ref{fig:#1}}
\newcommand{\SUBFIG}[1]{\subref{fig:#1}}

\newcommand{\TABLE}[1]{Table~\ref{tab:#1}}

\newcommand{\ITEM}[1]{(\ref{item:#1})}

%%% Textual modifiers
\newcommand{\s}[1]{``#1''}

%%% Common texts

\newcommand{\etal}{et al}
\newcommand{\ie}{i.e.}
\newcommand{\eg}{e.g.}
\newcommand{\etc}{etc}
\newcommand{\etcdots}{\etc\ldots}
\newcommand{\dash}{--}

%%%%%%%%%%%%%%%%%%%%%%%%%%%%%%%%%%%%%%%%%%%%%%%%%%%%%%%%%%%%%%%%%%%%%%%%%%%%%%%%
%% Math Environment Common Commands
%%%%%%%%%%%%%%%%%%%%%%%%%%%%%%%%%%%%%%%%%%%%%%%%%%%%%%%%%%%%%%%%%%%%%%%%%%%%%%%%

% Common operators
\DeclareMathOperator*{\argmin}{arg\,min}
\DeclareMathOperator*{\argmax}{arg\,max}
\newcommand{\abs}[1]{\left|\,#1\,\right|}
\newcommand{\dotp}[2]{\langle\,#1,#2\,\rangle}

% Common symbols
\newcommand{\Rs}{\mathbf{R}}
\newcommand{\Ss}{\mathcal{S}}
\newcommand{\Xs}{\mathcal{X}}
\newcommand{\Ys}{\mathcal{Y}}
\newcommand{\defeq}{\ensuremath{\mathrel{\stackrel{\mathrm{def}}{=}}}}

% Symbol modifiers
\newcommand{\vv}[1]{\boldsymbol{#1}}
\newcommand{\alt}[1]{\tilde{#1}}

% Text in Formula commands
\newcommand{\st}{\text{ s.t. }}


\nosubsubcounter

\title{Notes}
\author{Andrea Baisero}

\begin{document}
\maketitle

\section{Rules}

\subsection{Food and Winning}

Each player begins the game with $300(P-1)$ units of food, where $P$ is the
number of players.

If after any round you have zero food, you will die and no longer be allowed to
compete. All players who survive until the end of the game will receive the
survivor's prize.

The game can end in two ways. After a large number of rounds, there will be a
small chance each additional round that the game ends. Alternatively, if there
is only one person left with food then the game ends. In each case, the winner
is the person who has the most food when the game ends.

The game is designed to encourage both cooperation and competition between
players at different points in the game. Determining the right balance leads to
a winning strategy.

\subsection{Hunts}

Each round is divided into hunts. A hunt is a game played between you and one
other player. Each round you will have the opportunity to hunt with every other
remaining player, so you will have $P-1$ hunts per round, where $P$ is the
number of remaining players.

In a hunt you can make the decision to hunt (become a hunter) or feign sickness
and not hunt (become a slacker). Hunting is an active activity, so you will eat
$6$ units of food if you decide to hunt, and $2$ units of food if you decide not
to hunt. If both you and your partner do not hunt, then you get $0$ food from
the hunt. If only one of you decides to hunt then the hunt returns $6$ units of
food.  If both of you decided to hunt then the hunt returns $12$ units of food.
The total food return of the hunt is split evenly between you and your partner.
So, for example, if both of you decide to hunt, your individual net food
gain/loss from the hunt would be $12/2 - 6=0$ units, whereas if both of you
decided not to hunt your individual net food gain/loss is $0/2 - 2= -2$ food.

\subsection{Reputation}

Each player has a reputation $R$, which is defined as $R=\frac{H}{H+S}$, where
$H$ is the number of times you chose to hunt and $S$ is the number of times you
chose to slack (not hunt), both counted from the beginning of the game. Each
player begins with a reputation of $0$.

Each round, your tribe can save some time and hunt extra food if enough people
opt to hunt. Each round, a random integer m, with $0<m<P(P-1)$ will be chosen.
If the total number of hunters is greater than or equal to m for the upcoming
round, then everyone in the tribe gets $2(P-1)$ extra units of food after the
round, which means that on average everyone gets 2 units of food per hunt.
Before each round, you will find out the value of m.

Before each round you will be given: which number round you are in (from $1$ to
$\infty$), your food, your reputation, $m$, and an array with the reputation of
each remaining member of the tribe when the round started. The location of each
player in the array will be randomized after each round. You should decide your
moves for that round for each player you will play against, and return an array
of your decisions. After each hunt you will be given the food earned from each
hunt you played.

\subsection{Submission guidelines}

The logic of your code and what game theory analyses you apply for the various
steps must be either laid out as comments in the code or in a separate
document. The algorithm (and separate document, if any) should be submitted by
August 8, 2013 through your Brilliant account. Algorithms will be run
tournament-style to determine the winner. The finalist teams will then have a
short video interview with Brilliant staff, to verify that their work is their
own.

Prizes for algorithms eligible for the final game will be awarded at two
levels:

Grand Prize Winners \dash\ $5$ teams will receive the Grand Prize of \$1000.
The winning algorithm is guaranteed to receive the Grand Prize. $4$ other winners
will be selected based on the performance of their algorithm, logic, game
theory analysis, and presentation.  Finalists \dash\ Members of teams whose
algorithm survives the game will receive an \s{I survived} t-shirt.

\section{Recap of Rules}

\subsubsection{Annotation}

\begin{tabular}{|c|l|}
  \hline
  Symbol & Meaning \\
  \hline \hline
  $A_i$ & Agent i's action. $i = 0$ denotes my agent. \\
  \hline
  $R_i$ & Agent i's reputation. Approximates the probability of the opponent Hunting. \\
  \hline
  $N$ & number of rounds so far. \\
  \hline
  $H_i$ & Number of times agent i has hunted. Calculated as $\frac{H_i = R_i}{N}$. \\
  \hline
  $S_i$ & Number of times agent i has slacked. Calculated as $S_i = N - H_i$. \\
  \hline
  $W_0(A_0, A_i)$ & Reward function for my agent. First action always indicates
  my action. \\
  \hline
  $W_1(A_0, A_i)$ & Reward function for other agent. First action always indicates
  my action. \\
  \hline
\end{tabular}

\section{Strategy}

\subsection{Rewards}

\subsubsection{Reward Table}
The reward table describes the reward functions $W$.

\begin{figure}[h!]
  \centering
  \begin{equation*}
    W =\, 
      \begin{tabular}{|c||c|c|}
        \hline
        $A_0$ \textbackslash\ $A_i$ & H & S \\
        \hline\hline
        H & $0$ \textbackslash\ $0$ & $-3$ \textbackslash\ $+1$ \\
        \hline
        S & $+1$ \textbackslash\ $-3$ & $-2$ \textbackslash\ $-2$ \\
        \hline
      \end{tabular}
  \end{equation*}
\end{figure}

\subsubsection{Expected Rewards}

Let's try to analyse the expected rewards conditioned on the opponent's
reputation.

\begin{align*}
  E[W_0(A_0, A_i) / A_0 = H] &= E[W_0(H, A_i) \\
                             &= W_0(H, H) R_i + W_0(H, S) (1-R_i)\\
                             &= 0 R_i + -3 (1-R_i) \\
                             &= -3 + 3 R_i
\end{align*}
\begin{align*}
  E[W_0(A_0, A_i) / A_0 = S] &= E[W_0(S, A_i) \\
                             &= W_0(S, H) R_i + W_0(S, S) (1-R_i) \\
                             &= 1 R_i + -2 (1-R_i) \\
                             &= -2 + 3 R_i
\end{align*}
Notice how the second is always higher than the first. This simply confirms that
locally, the best action is to always Slack.

What if we also model our own action stochastically using our reputation?
\intodo{Come to think of it, the following is not very useful.. We don't have
uncertainty over our own action. I guess I was trying to model the opponent's
action depending on our own reputation, but it is badly done in this way. I'll
keep it just to keep track of all I've thought about, but ignore.}
\begin{align*}
  E[W_0(A_0, A_i)] &= E[W_0(A_0, A_i) / A_0 = H] R_0 + E[W_0(A_0, A_i) / A_0 = S] (1-R_0) \\
                   &= (-3 +3 R_i) R_0 + (-2 +3 R_i) (1-R_0) \\
                   &= -2 - R_0 + 3 R_i
\end{align*}
This result confirms the following:
\begin{inlist}
  \item Whenever we chose to slack rather than hunt, we gain $1$ reward point,
  \item Whenever the opponent chooses to hunt rather than to clask, we gain $3$
    reward points.
\end{inlist}

\subsubsection{Conclusion}
Choosing a policy which maximises the previous expectation leads to a static
policy of always slacking, which overall would lead to a consistent loss of 2
items of food for every turn.

However, we will see that maybe we are not really using all the information we
have. Up to now, we have assumed that the opponent's action is only dependent on
his responsibility, but we have left out that his actions in front of \emph{us}
are very likely to depend on our own responsibility.

The next section will try to analyse why and how our reputation changes our
opponents' actions, and how to make use of this to increase our expected food
value.

\subsection{Reputation Balancing}

\subsubsection{Why?}

We have already seen that slacking is the Nash Equilibrium solution, \ie\ the
local optimum when we try to maximize the expected rewards. Unfortunately, this
is far from a global optimum: Being this a prisoner dilemma problem, we have
that both players gain by cooperating, but only if they both cooperate. This is
the problem we are trying to address.

It is reasonable to assume that \emph{some} players will try to cooperate so
that they all gain by hunting together. If such a group forms, then I would
rather be in it than out of it. The intended goal is to avoid being classified
as a \s{slacker}: If you are a slacker, then everyone will slack against you.

The problem is that there is no means of communication with the community so as
to organize the up-coming of such a group: The only information we can exchange
is our reputation. The goal of reputation balancing should be to send a message
to your opponent: \s{look: you can trust me}.

On the other hand the creation of such a group opens ground for all sorts of
treason and backstabbing. Confirming yourself as a strong hunter is also not a
good strategy: If you are a compulsive hunter then other people will assume they
can slack their way into profit at your expenses. All in all, we just want to
keep our reputation just high enough for that to the message to pass. We gain
the maximum when we cooperate just enough to let the message pass. The question
now is: \emph{what is the threshold?}

One other interesting thing to consider is that the message we pass is only
effective if there is someone listening on the other end. If we are surrounded
by compulsive slackers, then there is little we can do but slack
ourselves\footnote{This is the reason why might want to keep a global reputation
measure at hand, but more on that later.}.

\intodo{Idea: Would it be a sensible strategy to make our own reputation
approach the global reputation? Possibly, but not always (\eg\ if we are
surrounded by strong hunters, or if we are head to head against just one last
player and we have more food than him).}

\subsubsection{What value to aim for?}

\subsubsection{How?}

\subsection{Rewards \dash\ reprise}

In this section we will try to find maximise or expected gain given not only the
opponent's reputation, but ours too, on the ground that both of them affect our
certainty of what the opponent's action might be.

\begin{align*}
  E[W_0(A_0, A_i) / R_0, R_i] &= ?
\end{align*}

\end{document}

